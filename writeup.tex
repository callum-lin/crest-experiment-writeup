\documentclass[14pt]{article}
\usepackage{graphicx} % Required for inserting images
\usepackage{appendix}

\title{The Performance of Water Rockets}
\author{Crest Silver Report}
\date{April 2025}

\usepackage{fontspec}
\setmainfont{Arial}
\usepackage{pgfgantt}
\usepackage{hyperref}
\hypersetup{
    colorlinks,
    citecolor=blue,
    filecolor=blue,
    linkcolor=blue,
    urlcolor=blue
}
\setlength{\parindent}{0pt}
\usepackage{parskip}
\setlength{\parskip}{1em}
\begin{document}
\pagenumbering{arabic}

\maketitle
\tableofcontents
\section{Planning the project}
\subsection{Our Aims and objectives}

The aim of this project was to investigate how changing the volume of water of water for a water rocket would affect the performance of the rocket.

We will consider our aim complete if we manage to:
\begin{itemize}
    \item Find the trend between the volume of water in a water rocket.
    \item Find the optimum volume of water for the water rocket to achieve maximum performance.
    \item Compare our results with our hypothesis and other sources.
    \item Explain the trend between the volume of water and performance of water rocket.
\end{itemize}
Our objectives are as follows:
\begin{enumerate}
    \item Choose suitable metrics for measuring the performance of water rockets.
    \item Design and carry out a suitable experiment at varying volumes of water.
    \item Analyse our data including trends, graphs and sources of error.
    \item Compare our results with our hypothesis and other articles and sources.
    \item Broadly discuss and explain our trend.
    \item Evaluate and reflect on our experiment, its wider implications and what we can take from it.
\end{enumerate}
\subsection{Wider Implications}
The flight of model water rockets is complex consisting of many interacting forces and equations including air pressure, hydrostatic pressure and thrust. Our experiment would be using the work of previous authors and extending it to another domain. We believe that a model would be very useful towards helping verify these complex interactions in a new niche -- though often unexplored environment.

Furthermore, our experiment is not only limited by its scope, as a water rocket produces thrust, this experiment is applicable to many other areas such as aerospace, spacecraft and planes which all work on the same principle as water rockets.

Our results are also important where they could be used to successfully validate models such as those used in the field of aerospace, where having real world data to validate and fit models provides confidence and the necessary data required to ensure maximum performance/efficiency characteristics of real world rockets.

We believe that a model that effectively simulates the flight of water rockets would be of interest in the classroom as well. While there are models online that already do this, they have not been tested with real data, or the data is not yet made available to the public. Having a water rocket simulator that is verified through real life testing provides greater assurance towards anybody learning about equations of motion.

A practical example of using our model would be to validate a similar rocket manufacturer's claim that ``In normal conditions, Water Rokit will fly to approximately 30 metres (100 feet) – about the height of a 10 storey building!''. \cite{1}
\subsection{Different Methods to Run the Project}
Before we started our project, we considered many varying approaches and thoroughly considered their pros and cons.

We had already narrowed down all our different approaches to completely independent work as our school was unable to allow us to shoot water rockets inside of school premises due to safety concerns.
Additionally due to the risk of damage to school property, we had to finance equipment and apparatus ourselves.

For our experiment regarding water rockets, our first course of action was to obtain said rocket, we considered the following approaches:
\subsubsection{Option 1: 3D Print own rocket at school/home}
\paragraph{Positives}
\begin{itemize}
    \item Allow for the greatest amount of flexibility with the ability to change prints later if required.
    \item Allowed for more experimental freedom as we would be able to easily replace parts.
    \item Allow for a more lightweight and cheap design
    \item Allow for experiments to be carried out in parallel as two copies would allow the team members to be able to conduct the experiment at the same time.
\end{itemize}
\paragraph{Disadvantages}
\begin{itemize}
    \item We would be directly responsible for making a design that worked and provided enough stability in the air, when others have already done that.

\end{itemize}
The advantages seemed to outweigh the disadvantages so after careful consideration we had started to plan for using a 3D printer instead.\\
However we faced a fatal flaw, we lacked the resources to obtain a rubber seal and hose at reasonable cost which would be fatal for our experiment -- thus we had no choice but to abandon option 1.
\subsubsection{Option 2: Buy a kit of Amazon}
\paragraph{Positives}
\begin{itemize}
    \item Premade kit so we knew our experiment would work.
    \item The aerodynamics and stability of the water rocket would be guaranteed.
    \item Cheaper than obtaining a 3D printer.
    \item Time investment woulld be less compared to option 1.
\end{itemize}
\paragraph{Negatives}
\begin{itemize}
    \item Less control over fine details of the rocket.
\end{itemize}
On the whole, while this approach has fewer advantages compared to option 1, we were satisfied with the advantages provided by this approach and we believed the negatives did not hold much weight. 
\\In the end, we went for this method.
\\\\
Now that the method of obtaining the rocket was sorted out, the measurement for determining the performance of the rocket was tested out.
\subsubsection{Option 1: Measure the height of the rocket}

This method would require us to launch rockets and measure the maximum height the rocket reaches -- using this as its measure of performance.
\paragraph{Positives}
\paragraph{}
This approach would be the most direct approach to measure the performance of the rocket --  in this case optimising for height reflects many real life rocket launches and model rocket launches were height is optimised.
\paragraph{Negatives}
\paragraph{}
While this approach initially appears to be the best approach, we found several issues. 
We thought about using string/twine attached to the rocket to accurately measure the height of the rocket, but this would be confounding as we would have to keep the string taut against the force of thrust which would be impossible to quantify. 

Furthermore, using twine for a distance of ~20 metres would cause damange to the twine and result in the twine knotting itself leading to a limited number of trail runs, which are again invalid due to the presence of the taut string acting against the force of friction.

We knew we could not use twine because of this, we knew it was impossible to measure with any degree of accuracy the final height of the rocket without buying expensive equipment such as an accelerometer and associated proprietary connection ports -- this option was off the table as:
\begin{itemize}
    \item We lacked the funds needed for even the cheapest accelerometers especially after knowing we would have to buy the water rocket kit already.
    \item We were unable to find a suitable mount point to the rocket even if we did manage to acquire an accelerometer as the nozzle was at the bottom, the sides would unbalance it and cause a torque force which we believed could not even be counteracted well enough by a static counterforce as it would still be in a state of unstable equilibrium
    \item We were unable to guarantee the safety of our equipment as without a parachute, the accelerometer would suffer very high impact forces. 
    \item Attaching an accelerometer would increase the momentum and therefore increase the impact force of the rocket when it fell to levels we felt unsafe about.
    \item We lacked any resources to attach/mount the accelerometer and we \\lacked prior experience with electronics hardware making the likelihood of catastrophic failure higher.
    \item Would require longer and more error prone analysis of data.

\end{itemize}
Even if we overcame these difficulties, the modification would increase our chances of failure and introduce a few more (mount point, parachute, data-analysis, electronics) uncertainties threatening our experiment. 

For this method, we did not foresee any way to measure the height without buying equipment (accelerometer or transmitter etc.); meaning in all circumstances, we believed the disadvantages heavily outweighed the advantages. 
\\
For these reasons above, we did not continue with option 1.
\subsubsection{Option 2: Measure the time taken for the rocket to fall}
This method is similar to option 1 with one main modification; timing the time the rocket is in the air the from launch to the landing.
\paragraph{Positives}
\begin{itemize}
    \item The independent variable, time, would be much easier to measure and time compared to say height or speed.
    \item Would not require any further investment of equipment other than the water rocket kit which we made already decided must be bought.
    \item Relatively straightforward compared to option 1.
\end{itemize}
\paragraph{Disadvantages}
\begin{itemize}
    \item This indirect measurement might not be what is normally correlated with the performance of said rocket.
    \item Would also require a lengthy, and perhaps more error prone, analysis of our raw results before they could be used.
\end{itemize}
We believed that option two struck a balance between the positives and negative aspects of this approach. It was very likely this method would succeed as it did not require any further manufacturer unsupported modifications of our water rocket.
\paragraph{Option 3: Launch the rocket horizontally and measure the final distance}
\paragraph{}
This approach is initially very attractive as it provides a set distance instead of time or velocity aligning better with real world approaches. This approach held many of the same advantages of option 2 with fewer disadvantages as distance would be easily measured however a fatal flaw came up in our calculations. At different volumes of water, the optimum angle to shoot it at would change and if we shot them horizontally, the frictional forces would be different. \\
Because there is no way to ensure the validity of our experiment, we could not continue with option 3 -- leading to option 2 being the best method here.
\subsection{Project Plan/Method}
For our method, we carefully considered the pros and cons of each method, cost was a prohibitive factor so we decided that buying a kit of Amazon would the best course of action as further explained above. Measuring the time taken for the rocket to launch would also be the best method of determining the performance of the rocket as it is more likely to succeed and would align better with out aims, allowing a good balance between positive and negative aspects of our experiment. We decided to use audio recordings mainly as this reduced human error as much as possible. 

\subsubsection{Practical Experiment}
Our experiment was set up as follows:
\begin{itemize}
    \item A 1L Coca-Cola bottle was stripped of its label with the cap partially broken off with a weight of 39g. see \cite{2}
    \item A paper scale of negligible mass was attached to the bottle for internal verification of volumes.
    \item A water bottle rocket kit was bought of Amazon.  \cite{3}
    \item A bicycle pump was bought of Amazon.
    \item The plastic cover of the water rocket kit was screwed onto the cap of the water bottle.
    \item Assuming the bottle is upright, the 3 fins were attached by sliding them into the slots up to down.
    \item The provided yellow plastic tubing was attached to the value fully.
\end{itemize}
The following approach was taken:
\begin{itemize}
    \item Known volumes of water was weighed out using either a measuring jug or balance.
    \item This water was poured into the bottle.
    \item The plastic seal was reattached to the bottle.
    \item The bottle was placed on a flat surface with the fins touching the ground.
    \item A video/audio recording was taken
    \item The pump was pumped.
    \item The pressure overcame the valve and the rocket shot up (and start a stopwatch if required)
    \item Listen to the rocket land (and stop the stopwatch if required.
    \item End the video/audio recording.
\end{itemize}
\subsubsection{After the Experiment}
Analysis of raw data:
\begin{itemize}
    \item If required, videos and audio was analysed by sound/movement to determine the time the rocket was in the air.
    \item Graphs and conclusions were drawn out
\end{itemize}
Creating the model:
\begin{itemize}
    \item The water rocket attributes were taken except pressure which could not be experimentally tested.
    \item Research was conducted into the equations of motion.
    \item A model was made.
    \item The model was fitted with experimental data to get the constant pressure at which the valve fails.
\end{itemize}
After the model:
\begin{itemize}
    \item Explain out results.
    \item Draw conclusions.
    \item Use our model to test its wider implications.
\end{itemize}
Our rationale for creating a model would be to effectively increase our trust in our results as that way, we can see how the best line curve correlates with the equations governing the laws of motion.

Furthermore, using this method would provide an adequate level of difficulty allowing us to test out skills benefiting out knowledge of the sciences. We also believe that using a model would help increase the impact of our work.
\subsection{Plan and Timeline}
\section{Throughout the project}
\subsection{Materials, Resources and people}
\subsection{Background Information}
\section{Finalising the project}
\subsection{Conclusions}
\subsection{How our actions impacted the project}
\subsection{What we learned and further improvements to our experiment}
\section{Project-wide criteria}
\subsection{Underlying physics}
\subsubsection{Propulsion and Motion}
As air is pumped into the rocket from the pump, the internal pressure increases. Eventually, the pressure will be great enough to overcome the resistive force of the nozzle, due to pressure acting in all directions, which results in the launch of the rocket. Immediately, the water will be expelled downwards due to the high pressure, thus providing an equal and upwards thrust which will propel the rocket until all of the water is expelled. The rocket will then continue to rise until it reaches a velocity of 0 at the peak of its trajectory and it will then fall back down. The time taken for the rocket to fall after launch can be measured, because the time taken for it to fall will be proportional to the initial acceleration of the rocket, and this quantity serves as a measure of performance. 

The aim of finding the optimal water to air ratio is based on the principle that too much water will make the rocket heavier, while too little water means reduced reaction mass to generate thrust. The optimal ratio will produce the greatest acceleration and therefore the greatest recorded time. 
\subsubsection{Aerodynamics}
There are small disturbances in the air which may push a flying bottle of its straight path, which is why fins are required to stabilise the rocket and prevent it from veering off course. The fins work by producing a restoring force, which will counteract changes in pitch or rotation. This happens because ...
The main reason for the fins being placed at the end of the bottle is to maximise torque, which results in a greater stabilising force.
\begin{thebibliography}{0}
\bibitem{1}
Frequently Asked Questions [Internet]. Hertford (U.K.): HINTERLAND LIMITED; c2025. [Updated 2025; cited 2025 Mar 22]; [First answer under General Frequently Asked Questions]: Available from https://waterrokit.com/frequently-asked-questions/\#elementor-tab-content-1001

\bibitem{2}
Coke-1-litre | taste thirst [internet]. London (U.K.): Taste Thirst c2023. [Updated 2023; cited 2025 Mar 22]; [Image of Coke-1 litre bottle]: Available from: https://tastethirst.com/product/coke-1litre/
\bibitem{3}
Funtime Gifts 10657 Bottle Rocket Kit : Amazon.co.uk: Toys \& Games. Seattle (U.S.) c2025. [Updated 2025; cited; 2025 Mar 22]; [Product Listing]: Available from https://amzn.eu/d/4BXIgWf
[add references later cause they are pretty difficult.]
\end{thebibliography}

\section{Appendices}
\subsection{Appendix A}
\label {Appendix A}

\end{document}
